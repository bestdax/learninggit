\section{Git Fsck}

\shin{git fsck} 是 \shin{Git} 中的一个命令,用于检查 \shin{Git} 存储库的完整性。它会扫描 \shin{Git} 仓库中的所有对象(如提交对象、树对象、标签对象等),并检查是否存在损坏、丢失或其他异常。该命令主要用于诊断和修复潜在的问题。

\subsection*{主要功能}
\begin{itemize}
	\item \textbf{验证对象完整性}:\shin{git fsck} 会检查所有的对象是否存在,且是否符合预期的哈希值。
	\item \textbf{查找悬空对象}:它可以找到未被引用的对象,这些对象可能是因为重写历史或其他原因而丢失的提交。
	\item \textbf{报告问题}:如果存储库中的任何对象出现问题,\shin{git fsck} 会列出具体的错误信息和可能的问题。
\end{itemize}

\subsection*{常用选项}
\begin{itemize}
	\item \texttt{--full}:执行全面的检查。默认情况下,\shin{git fsck} 只进行轻量级检查,但使用此选项可以执行更深入的检查。
	\item \texttt{--unreachable}:显示无法访问的对象,即那些未被任何引用(如分支、标签)引用的对象。
	\item \texttt{--no-dangling}:不报告悬空的对象。
	\item \texttt{--lost-found}:对于找不到引用的对象,\shin{Git} 会将它们放在\\ \shin{.git/lost-found} 目录下。
\end{itemize}

\subsection*{使用示例}
\begin{shellcmd}
git fsck
\end{shellcmd}

运行上面的命令后,\shin{Git} 会扫描整个存储库并输出任何找到的问题。

\subsection*{应用场景}
\begin{itemize}
	\item \textbf{诊断存储库问题}:当怀疑存储库损坏或数据丢失时,可以使用\\ \shin{git fsck} 来检查存储库的健康状态。
	\item \textbf{恢复丢失的提交}:如果一些提交被意外丢失,可以通过 \shin{git fsck} 找到悬空的对象并进行恢复。
\end{itemize}

总的来说,\shin{git fsck} 是一个非常有用的工具,可以帮助你维护 \shin{Git} 存储库的健康和完整性。

\subsection*{实际案例}
我在把代码添加到暂存区后,发现有一个编译的文件夹也被添加进去了,然后进行了一次误操作,用了\shin{git reset
	--hard},这下退回到之前的提交,但是新增加的内容丢失了。幸好用\shin{git fsck --lost-found}查找到悬空对象,然后用\shin{git show <blob
	hash>}逐个查找到丢失的文件。最后用 \shin{git show <blob hash> > filename}成功将文件恢复。
