\section{Git Remote}
\begin{shellcmd}
git remote add <origin> <URL>
git remote prune <origin> # 删除远程仓库在不存在的分支
\end{shellcmd}

不知道是不是应该写在这个章节里面,最近对Git本地的仓库做了两件大事:

\begin{enumerate}
	\item 将Rime仓库一个仓库推送到两个远程(public的部分把隐私的信息去除)
	\item 将dotfiles仓库中的nvim配置及ohmyzsh的配置分离出来
\end{enumerate}

下面分别来将这两个过程记录一下。

\subsection{一对多推送}
这个过程需要解决从原来的主分支建立一个全新孤立分支,因为我的隐私信息是从第一次提交就有的。因为以前对Git还不了解,觉得以前的历史没有什么用,当时是把\shin{.git}文件夹删除了以后重新初始化的。
\begin{shellcmd}
	git checkout -b <new_branch_name> --orphan <commit>
\end{shellcmd}
这个命令会创建一个名为\shin{<new_branch_name>}的分支,内容是基于\\\shin{<commit>},然后将\shin{<new_branch_name>}作为它的新分支。这时这个分支上的提交记录是空白的,所有的文件都存放在暂存区,把隐私文件从暂存区里面推出,然后删除掉隐私信息再加入到暂存区,然后提交就可以了。

下一步在远程创建一个新的仓库,并在本地添加进来,用下面的命令:
\begin{shellcmd}
git remote add <origin> <URL>
git push -u <origin>/main <new_branch_name>
\end{shellcmd}
就这样,一个本地仓库就可以对应多个远程仓库了。

\subsection{拆分仓库}
\begin{enumerate}
	\item 准备工具
	      \begin{shellcmd}
		brew install git-filter-repo
		\end{shellcmd}
	\item 克隆旧仓库到新的目录
	      \begin{shellcmd}
		git clone <old_repo_url> <new_repo_dir>
		\end{shellcmd}
	\item 拆分
	      \begin{shellcmd}
		git repo-filter --subdirectory-filter <dir>
		# 将<dir>以外的内容过滤掉
		\end{shellcmd}
	\item 新建一个远程仓库,并添加到过滤后的新仓库,推送即可
	\item 将本地的旧仓库去除过滤的目录及相关的提交
	      \begin{shellcmd}
		git filter-repo --path <filtered-dir> --invert-paths
		\end{shellcmd}
	\item 删除远程仓库,新建一个同名的远程仓库再添加到更新后的本地旧仓库,推送即可
\end{enumerate}
